To identify the effectiveness of national cybersecurity policies,
it is essential to analyze the correlation between the implementation of these policies and the subsequent trends in cybercrime.
By examining the distribution of cybercrimes and comparing it with the timing and content of various national policies,
we can discern patterns that highlight which measures are particularly effective or ineffective.
This analysis will focus on key metrics such as the reduction in
cybercrime incidents, the success rate of prosecutions, and the overall resilience of national cybersecurity infrastructures.
Through this data-driven approach,
we aim to provide actionable insights for the development and refinement of cybersecurity policies.
\subsection{Selection of Representative Centroid Countries}\label{subsec:selection-of-representative centroid-countries} %4.1
Having constructed a clustering model to categorize countries into five clusters (T1 to T5) based on GCI and other relevant metrics,
we now proceed to analyze the effectiveness of cybersecurity policies within each cluster.
To ensure a representative and data-driven analysis,
we will select one central country from each cluster that meets the following criteria:
\begin{itemize}
    \item \textbf{Representativeness:}
    The selected country should typify the overall characteristics of its cluster,
    reflecting the general trends and patterns observed within that group.
    \item \textbf{Data Availability:}
    The country must have sufficient historical data on cybersecurity policies and legislation enacted over the past two decades,
    allowing for a comprehensive analysis of policy impacts.
\end{itemize}
By focusing on these representative countries,
we aim to draw meaningful insights into the effectiveness of various cybersecurity policies and laws,
which can then be generalized to other countries within the same cluster.

To identify the representative country for each cluster,
we first calculate the average Global Cybersecurity Index (GCI) for each cluster.
The average GCI, denoted as \(\overline{GCI}\), is computed as follows:
\[ \overline{GCI} = \frac{\sum_{i=1}^{n} GCI_i}{n} \]
where \(n\) is the number of countries in the cluster.
Next, we compute the absolute deviation of each country's GCI from the cluster average:
\[|GCI_i - \overline{GCI}|\] .
The country with the smallest deviation is considered the most representative of its cluster.
After this initial selection, we further filter out countries with insufficient or incomplete legal and policy documentation.

Through this process, we identify the following representative countries for each cluster:
\begin{itemize}
    \item \textbf{T1: United States}
    ~\cite{
        congress-website,
        nist-website,
        dhs-website,
        sec-website,
        whitehouse-website,
        investigatory-powers-act-2016,
        ncsc-uk,
        telecom-security-act-2021,
        uk-cyber-security-requirements-2024,
        uk-cybersecurity-timeline-2024}
    \item \textbf{T2: Japan}
    ~\cite{
        it-basic-law-japan,
        ppc-legal-japan,
        nisc-japan,
        mofa-japan,
        japan-law-translation,
        cs-strategy-2015-japan,
        cs-strategy-2018-japan,
        telecom-business-act-japan,
        cs-strategy-2021-japan}
    \item \textbf{T3: China}
    ~\cite{
        international-cybercrime,
        cybersecurity-law-china,
        internet-censorship-china,
        china-data-security-regulations,
        cryptography-law-china}
    \item \textbf{T4: Costa Rica}
    ~\cite{
        costa-rica-cybersecurity-strategy,
        costa-rica-pop-up}
    \item \textbf{T5: Namibia}
    ~\cite{
        namibia-pop-up,
        namibia-digital-odyssey,
        namibia-cybersecurity-strategy}
\end{itemize}
