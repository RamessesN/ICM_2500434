%! Author = Yilin
%! Date = 2025/1/27

\subsection{Symbol Description}\label{subsec:symbol-description} %2.1
\begin{tabular}{cl}
    \hline
    \textbf{Symbol} & \textbf{Description} \\
    \hline
    $D_i$                     & Cybercrime distribution in each country. \\
    $P_i$                     & Population of each country. \\
    $y$                       & Transformed value using \( y = \log(1 + x) \). \\
    $\mathbb{E}[Crime_t]$     & Poisson regression model predicted value. \\
    $\beta_0$                 & Intercept term. \\
    $\beta_1,\beta_2,\beta_3$ & Regression coefficients. \\
    $Bill_{t-k}$              & Lagged variable. \\
    $K$                       & Greatest lagged number. \\
    $\rho$                    & Spearman rank correlation coefficient \\
    $d_i$                     & Difference between the ranks of each pair of observations \\
    $n$                       & Number of observations \\
    \hline
\end{tabular}

\bigskip

\noindent
\begin{tabular}{cl}
    \hline
    \textbf{Abbreviation} & \textbf{Full Form} \\
    \hline
    GCI   & Global Cybersecurity Index\cite{gci-2024} \\
    VERIS & the Vocabulary for Event Recording and Incident Sharing\cite{veris} \\
    VCDB  & the VERIS Community Database\cite{vcdb} \\
    GDP   & Gross Domestic Product \\
    VIF   & Variance Inflation Factor \\
    \hline
\end{tabular}
\subsection{Assumptions}\label{subsec:assumptions} %2.2
\begin{itemize}
    \item Countries with a population below 5\% are excluded from the consideration of cybercrime distribution because
        a small change of number could bring a significant difference to statistical analysis.
    \item In the statistical analysis of the global distribution of cybercrime,
        factors such as national population growth, internet access, wealth levels, and education levels
        are assumed to have no significant impact on the quantitative distribution of cybercrime incidents.
        This study hypothesizes that the distribution of cybercrime can be more intuitively understood by focusing solely on the number of incidents,
        independent of these socio-economic variables.
        This assumption is based on all available data recorded since the inception of cybercrime statistics,
        aiming to isolate the distribution patterns of cybercrime from other potential influencing factors.
    \item The impact of newly enacted laws or policies on cybercrime is not instantaneous;
        there is a time lag before their effects become evident.
\end{itemize}
