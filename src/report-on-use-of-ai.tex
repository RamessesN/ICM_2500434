%! Author = Yilin
%! Date = 2025/1/27

\begin{enumerate}
	% 1. United States
    \item \textbf{OpenAI ChatGPT( Jan 25, 2025 version, ChatGPT-4o(Internet Search) )}
    \begin{description}
        \item Query1: Please give me more US cybersecurity laws and regulations from 2000 to 2023.
        \item Output: Here is a list of significant U.S.\ cybersecurity laws and regulations from 2000 to 2023:
        \begin{enumerate}[label=\arabic*.]
        		\item USA PATRIOT Act (2001)
			\begin{itemize}
				\item Expanded law enforcement’s ability to surveil and intercept communications for counterterrorism purposes, including provisions affecting cybersecurity.
			\end{itemize}
			\item Federal Information Security Management Act (FISMA) (2002)
			\begin{itemize}
				\item Required federal agencies to develop, document, and implement programs to secure information and systems.
			\end{itemize}
			\item Cybersecurity Research and Development Act (2002)
			\begin{itemize}
				\item Funded research and development for cybersecurity tools and technologies through the National Science Foundation (NSF) and National Institute of Standards and Technology (NIST).
			\end{itemize}
			\item E-Government Act (2002)
			\begin{itemize}
				\item Established guidelines for improving electronic government services, including the management of information security.
			\end{itemize}
			\item Homeland Security Act (2002)
			\begin{itemize}
				\item Created the Department of Homeland Security (DHS) and included provisions for critical infrastructure protection against cyber threats.
			\end{itemize}
			\item Presidential Policy Directive 21 (PPD-21) (2013)
			\begin{itemize}
				\item Aimed to strengthen critical infrastructure security and resilience, emphasizing information sharing across sectors.
			\end{itemize}
			\item National Institute of Standards and Technology (NIST) Cybersecurity Framework (2014)
			\begin{itemize}
				\item Voluntary guidelines for private and public organizations to manage and reduce cybersecurity risks.
			\end{itemize}
			\item Cybersecurity Information Sharing Act (CISA) (2015)
			\begin{itemize}
				\item Enabled the sharing of cyber threat information between private entities and the federal government.
			\end{itemize}
			\item Federal Cybersecurity Enhancement Act (2015)
			\begin{itemize}
				\item Mandated the use of intrusion detection and prevention systems (e.g., EINSTEIN) on federal networks.
			\end{itemize}
			\item National Defense Authorization Act (NDAA) Provisions (2019)
			\begin{itemize}
				\item Introduced supply chain security measures and banned federal agencies from using certain Chinese-made telecommunications equipment.
			\end{itemize}
			\item CLOUD Act (2018)
			\begin{itemize}
				\item Clarified legal processes for accessing electronic data stored abroad while balancing privacy and law enforcement interests.
			\end{itemize}
			\item SEC Guidelines on Cybersecurity Disclosure (2018, Updated 2023)
			\begin{itemize}
				\item Required public companies to disclose cybersecurity risks, incidents, and governance strategies to protect investors.
			\end{itemize}
			\item Internet of Things (IoT) Cybersecurity Improvement Act (2020)
			\begin{itemize}
				\item Set baseline security standards for IoT devices used by federal agencies.
			\end{itemize}
			\item Cyber Incident Reporting for Critical Infrastructure Act (CIRCIA) (2022)
			\begin{itemize}
				\item Required critical infrastructure operators to report significant cyber incidents within 72 hours and ransomware payments within 24 hours.
			\end{itemize}
			\item Quantum Computing Cybersecurity Preparedness Act (2022)
			\begin{itemize}
				\item Directed federal agencies to migrate to quantum-resistant cryptographic standards.
			\end{itemize}
			\item Strengthening American Cybersecurity Act (2022)
			\begin{itemize}
				\item Consolidated multiple cyber laws, introduced new rules for incident reporting, and enhanced federal cybersecurity protections.
			\end{itemize}
			\item CHIPS and Science Act (2022)
			\begin{itemize}
				\item Included cybersecurity measures for semiconductor manufacturing and technology development.
			\end{itemize}
			\item Telecommunications Supply Chain Security (2023)
			\begin{itemize}
				\item Expanded restrictions on telecommunications companies deemed national security risks.
			\end{itemize}
        \end{enumerate} \\
    \end{description}

	% 2. Japan
    \item \textbf{OpenAI ChatGPT( Jan 25, 2025 version, ChatGPT-4o(Internet Search) )}
    \begin{description}
		\item Query2: Also give me the laws and regulations related to network security in Japan from 2000 to the present.
		\item Output: Here are Japan’s key cybersecurity-related laws and policies from 2000 to the present,:
        \begin{enumerate}[label=\arabic*.]
        		\item Basic Act on the Formation of Advanced Information and Telecommunications Network Society (2000)
			\begin{itemize}
				\item Established a framework for building advanced information and telecommunications society in collaboration with public and private sectors.
			\end{itemize}
			\item Act on the Protection of Personal Information (2003)
			\begin{itemize}
				\item Japan’s first law focusing on personal information protection, setting rules for the collection, use, and management of personal data.
			\end{itemize}
			\item Comprehensive Strategy on Information Security (2003)
			\begin{itemize}
				\item Introduced by the Ministry of Economy, Trade, and Industry (METI), emphasizing the concept of a “society assuming accidents,” aiming to prevent and mitigate cybersecurity incidents.
			\end{itemize}
			\item Basic Plan on Information Security (2006)
			\begin{itemize}
				\item Proposed a “new public-private partnership model,” clarifying roles and cooperation between the government and private sector in ensuring information security.
			\end{itemize}
			\item Cybersecurity Strategy (2013)
			\begin{itemize}
				\item Japan’s first cybersecurity strategy, addressing comprehensive challenges in cyberspace and advocating for international collaboration.
			\end{itemize}
			\item Basic Act on Cybersecurity (2014)
			\begin{itemize}
				\item Passed in November 2014, it defined the responsibilities of governmental, private, and academic institutions in promoting cybersecurity and established the Cybersecurity Strategic Headquarters.
			\end{itemize}
			\item Cybersecurity Strategy (2015)
			\begin{itemize}
				\item Highlighted Japan’s vision of ensuring “a free, fair, and secure cyberspace” and aligned with international cybersecurity standards.
			\end{itemize}
			\item Cybersecurity Strategy (2018)
			\begin{itemize}
				\item Updated the 2015 strategy by including concrete action plans, enhancing organizational frameworks, and strengthening international cooperation.
			\end{itemize}
			\item Telecommunications Business Act Amendments (2019)
			\begin{itemize}
				\item Introduced measures to ensure the security of telecommunications infrastructure, including vendor restrictions for critical infrastructure.
			\end{itemize}
			\item Cybersecurity Strategy (2021)
			\begin{itemize}
				\item Revised strategy to align with the latest global cybersecurity trends, focusing on emerging technologies such as IoT and 5G, and reinforcing resilience against cyber threats.
			\end{itemize}
        \end{enumerate}
    \end{description}

	% 3.China
    \item \textbf{OpenAI ChatGPT( Jan 25, 2025 version, ChatGPT-4o(Internet Search) )}
    \begin{description}
        \item Query3: So do China.
		\item Output: Since 2000, China has developed a comprehensive framework of cybersecurity laws and regulations to address the evolving digital landscape.
			Here are some key legislations and policies:
        \begin{enumerate}[label=\arabic*.]
        		\item Decision on Internet Security (2000)
			\begin{itemize}
				\item In 2000, the Standing Committee of the National People’s Congress passed the “Decision on Internet Security,” establishing foundational principles for internet regulation and security in China.
			\end{itemize}
			\item Cybersecurity Law of the People’s Republic of China (2017)
			\begin{itemize}
				\item Enacted by the Standing Committee of the National People’s Congress on November 7, 2016, and implemented on June 1, 2017, this law aims to increase data protection, data localization, and cybersecurity in the interest of national security.
			\end{itemize}
			\item Data Security Law of the People’s Republic of China (2021)
			\begin{itemize}
				\item Promulgated on June 10, 2021, and effective from September 1, 2021, this law establishes a data classification framework based on national security principles, imposing data localization requirements and mandating that data transfers to foreign law enforcement or judicial agencies require official approval.
			\end{itemize}
			\item Personal Information Protection Law of the People’s Republic of China (2021)
			\begin{itemize}
				\item Effective from November 1, 2021, this law is China’s first comprehensive legislation on personal data rights, modeled after the European Union’s General Data Protection Regulation (GDPR), outlining requirements for data collection, processing, and protection.
			\end{itemize}
			\item Regulations on Network Data Security Management (2023)
			\begin{itemize}
				\item Issued in 2023, these regulations implement the Cybersecurity Law, Data Security Law, and Personal Information Protection Law, providing practical guidance on compliance requirements for businesses operating in China.
			\end{itemize}
        \end{enumerate}
    \end{description}

	% 4. Costa Rica
    \item \textbf{OpenAI ChatGPT( Jan 25, 2025 version, ChatGPT-4o(Internet Search) )}
    \begin{description}
		\item Query4: So do Costa Rica.
		\item Output: Costa Rica has developed a comprehensive legal framework to address cybersecurity and data protection since 2000.
			Key laws and policies include:
        \begin{enumerate}[label=\arabic*.]
        		\item Law on Protection of the Person Against the Processing of Personal Data (Law No. 8968) (2011)
			\begin{itemize}
				\item Enacted in 2011, this law regulates the processing of personal data to protect individuals’ privacy rights.
					It establishes guidelines for data collection, storage, and usage, and led to the creation of the Agency for the Protection of Inhabitants’ Data (PRODHAB) to oversee compliance.
			\end{itemize}
			\item	Law on Computer-Related Crimes (Law No. 9048) (2012)
			\begin{itemize}
				\item Implemented in 2012, this law amends the Criminal Code to define and penalize various cybercrimes, including unauthorized access, data interference, and system interference.
					It provides a legal basis for prosecuting offenses committed through information and communication technologies.
			\end{itemize}
			\item National Cybersecurity Strategy (2017)
			\begin{itemize}
				\item In 2017, the Ministry of Science, Technology, and Telecommunications (MICITT) introduced the National Cybersecurity Strategy to enhance the country’s cybersecurity posture.
					The strategy outlines objectives such as strengthening legal frameworks, promoting public-private collaboration, and building cybersecurity capabilities.
			\end{itemize}
			\item Accession to the Budapest Convention (2017)
			\begin{itemize}
				\item Costa Rica acceded to the Council of Europe’s Convention on Cybercrime, known as the Budapest Convention, in 2017.
					This accession aligns the country’s legal framework with international standards for combating cybercrime and facilitates international cooperation in cybercrime investigations.
			\end{itemize}
        \end{enumerate}
    \end{description}

	% 5. Canada
    \item \textbf{OpenAI ChatGPT( Jan 25, 2025 version, ChatGPT-4o(Internet Search) )}
    \begin{description}
		\item Query5: So do Canada.
		\item Output:
        \begin{enumerate}[label=\arabic*.]
        		\item Personal Information Protection and Electronic Documents Act (PIPEDA) 2000
			\begin{itemize}
				\item PIPEDA, enacted in 2000, is Canada’s primary legislation governing the collection, use, and disclosure of personal information in the private sector.
					It includes provisions for cybersecurity by requiring organizations to protect personal data and report any breaches that affect individuals’ privacy.
			\end{itemize}
			\item The Anti-terrorism Act (2001)
			\begin{itemize}
				\item This act was passed shortly after the 9/11 attacks and enhanced the government’s ability to combat terrorism, including cyberterrorism.
					It provided law enforcement agencies with broader powers to monitor communications and intercept data related to national security threats.
			\end{itemize}
			\item Public Safety Canada’s Cybersecurity Strategy (2009)
			\begin{itemize}
				\item This strategy, launched in 2009, aimed to strengthen Canada’s cybersecurity resilience.
					It involved collaboration between government, industry, and international partners to secure Canada’s critical infrastructure from cyber threats.
			\end{itemize}
			\item The National Cyber Security Strategy (2010)
			\begin{itemize}
				\item Canada’s 2010 cybersecurity strategy outlined key priorities such as protecting critical infrastructure, enhancing cybercrime law enforcement, and promoting international cybersecurity collaboration. 
			\end{itemize}
			\item The Cybersecurity Act (2014)
			\begin{itemize}
				\item While primarily a policy rather than formal legislation, this document was a key milestone in Canada’s approach to cybersecurity.
					It focused on the protection of Canadian businesses, citizens, and government data from increasing cyber threats.
			\end{itemize}
			\item The Digital Privacy Act (2015)
			\begin{itemize}
				\item This act amended PIPEDA to address the rising concerns over cybersecurity, particularly regarding data breach notification requirements.
					Organizations are now legally required to report breaches that pose a risk of significant harm to individuals.
			\end{itemize}
			\item The Critical Infrastructure Protection Act (2017)
			\begin{itemize}
				\item This law aims to enhance the cybersecurity protections around Canada’s critical infrastructure, including sectors like energy, telecommunications, and transportation, by fostering cooperation between the government and private sector.
			\end{itemize}
			\item Cybersecurity Strategy for Canada (2018)
			\begin{itemize}
				\item This comprehensive strategy sought to improve cybersecurity across Canada by focusing on strengthening the country’s defense against cybercrime, supporting innovation, and protecting critical infrastructure.
					It also emphasized Canada’s role in international cybersecurity collaboration.
			\end{itemize}
			\item The Bill C-11: Digital Charter Implementation Act (2020)
			\begin{itemize}
				\item This bill proposed a major overhaul of Canadian privacy law, introducing stricter regulations on how personal data is handled and protecting citizens’ data online.
					It includes provisions to enhance cybersecurity protections.
			\end{itemize}
			\item Canada’s National Cybersecurity Strategy (2022)
			\begin{itemize}
				\item In 2022, Canada launched its National Cybersecurity Strategy to address the growing threat of cyberattacks.
					The strategy outlines objectives such as improving government security, fostering international partnerships, and advancing cybersecurity awareness and research.
			\end{itemize}
			\item The Telecommunications and Cybersecurity Laws (2021)
			\begin{itemize}
				\item This policy introduced new cybersecurity measures to ensure that Canada’s telecommunications sector is better protected from foreign threats.
					It includes provisions for private companies to take on increased cybersecurity responsibilities.
			\end{itemize}
			\item The National Cybersecurity Innovation and Research Strategy (2023)
			\begin{itemize}
				\item This strategy focuses on fostering innovation in cybersecurity technologies and enhancing collaboration with private, public, and international stakeholders to better address emerging cyber threats.
			\end{itemize}
        \end{enumerate}
    \end{description}
\end{enumerate}