%! Author = Yilin
%! Date = 2025/1/27

\subsection{As for the 1^{st} Question} %7.1
    By filtering the data and drawing the global heat map of cybercrime distribution,
    the global distribution of cybercrime can be visualized.
    Then, by filtering the data according to specific indicators
    (Confirmed, Near miss, Reported, Prosecuted)
    and drawing the corresponding global heat map,
    the distribution of these indicators can be visualized.

    Pure data screening and analysis need to be based on a huge amount of data.
    For these big countries in Europe and the United States, the annual amount of data is very considerable,
    so the effect is very prominent.
    But for some small countries in Africa or Asia,
    the difference in the distribution map of cybercrime is not very obvious because of the insufficient amount of data.

\subsection{As for the 2^{nd} Question} %7.2
    \begin{enumerate}
        \item Through processing the filtered data and drawing a line chart,
            it can be more intuitive to see which laws and policies have benefits for the national cybercrime rate control;
            However, the timeliness of legal policies cannot be rationally judged, that is,
            whether the law has an impact on the future cannot be directly considered.
        \item Introduction of Poisson model: Because the number of cybercrimes is a typical non-negative integer data,
            Poisson regression analysis can avoid the continuity assumption of ordinary linear regression,
            and can model and explain the nonlinear association between the number of bills and the crime rate and zero inflation data.
            Furthermore, the regression coefficients can be used for policy evaluation by calculating the incidence ratio $IRR$ .
            Over-discretization of the data can lead to bias in the model,
            and the model will only work for relationships that are roughly linear,
            and it will be less useful if the actual relationship is more complex (e.g., if there are interaction effects).
        \item Introduction of time hysteresis effect:
            The introduction of time lag effect can reflect the delay of the policy coming into effect, that is,
            the release of legal policies does not necessarily produce benefits at the time but will have potential impacts on the future.
            Then, the optimal lag period was selected by cross validation,
            and the key time node for the policy to be effective was clarified.
            But it ignores the immediate effect on the law.
    \end{enumerate}
    Therefore, it is necessary to combine the data analysis and Poisson \& time lag effect model
    to obtain the income effect of legal policies on the national Internet crime rate.

