%! Author = Yilin
%! Date = 2025/1/26

\subsection{Background}\label{subsec:background} %1.1
	In the digital age, the speed and scale of global connectivity have reached unprecedented heights.
	However, with technological advancements, \textbf{cybercrime} has also become increasingly complex and diverse.
	These crimes pose significant threats and challenges to personal privacy, corporate assets, national security, and social stability.
	The transnational and covert nature of cybercrime makes it difficult to address effectively.
	Attackers often exploit legal differences and technical vulnerabilities across countries to evade accountability.
	Additionally, many businesses and institutions, concerned about their reputation and commercial interests,
	often choose \textbf{not to publicly report} cyberattacks, opting instead to pay ransoms or handle incidents privately.
	This further exacerbates the hidden nature of cybercrime.
	Developed countries, with their highly digitized economic and social structures, are often prime targets for cybercrime, while
	developing countries face their own unique challenges.

	To address this global challenge, countries have introduced national cybersecurity \textbf{policies}
	aimed at enhancing their defensive capabilities through legal, technical, and organizational cooperation.
	The \textbf{effectiveness} of these policies varies significantly across nations, and these differences may be closely related to factors
	such as policy design, enforcement, technological infrastructure, education levels, economic development, and internet penetration rates.
	In this context, understanding which factors make certain countries' cybersecurity policies more effective has become a critical issue.
	By analyzing the global distribution of cybercrime, national cybersecurity policies, and their outcomes,
	we can identify which policies and laws are particularly effective in preventing, prosecuting, and mitigating cybercrime.
	This data-driven analysis not only helps countries improve their cybersecurity policies
	but also provides valuable insights for global cybersecurity cooperation.
\subsection{Restatement of the problem}\label{subsec:restatement-of-the-problem} %1.2
	We are required to identify patterns
	that can inform the data-driven development and refinement of national cybersecurity policies and laws,
	which focused on those that have demonstrated effectiveness.
	Our goal is to develop a theory explaining what constitutes a strong national cybersecurity policy and then
	support it with a data-driven analysis.
	Several key aspects are listed below:
	\begin{itemize}
		\item Patterns in cybercrime distributed worldwide.
		\item Assessment to effectiveness of National Cybersecurity Policies.
		\item Correlation between national demographics and our distribution analysis.
	\end{itemize}
\subsection{Analysis of problems}\label{subsec:analysis-of-problems} %1.3
	To address the problem requirements, our analysis focuses on three core aspects:
	identifying global patterns in cybercrime distribution,
	analyzing the effectiveness of national cybersecurity policies through comparative and time-sensitive evaluations, and
	exploring the correlation between cybercrime trends and national demographic factors.
	This structured approach aims to inform the data-driven development and refinement of robust cybersecurity policies and laws
	by leveraging insights from global data and empirical analysis.
	\subsubsection{Cybercrime distribution across the world} %1.3.1
		\begin{itemize}
			\item Process the JSON files published on VCDB .
			\item Use GCI as the primary indicator to assess countries with disproportionately high levels of cybercrime.
			\item Group countries based on GCI Tier measure, and
				visually represent the distribution of cybercrime in each group using heatmaps.
		\end{itemize}
	\subsubsection{Effective policy or law analytical approach} %1.3.2
		\begin{itemize}
			\item Identify a representative country from each of the T1 to T5 national clusters and
				collect cybersecurity-related policies enacted by these countries.
			\item Plot time-series line graphs depicting the trends of cybercrime over time and
				analyze which policies have been effective in curbing criminal activities.
			\item Conduct a targeted analysis of specific policy models' effectiveness in certain countries,
				focusing on particular indicators.
			\item Integrate temporal factors and national contexts to comprehensively evaluate the impact of policies.
		\end{itemize}
	\subsubsection{National demographics correlation} %1.3.3
		\begin{itemize}
			\item For each country, preprocess the annual proportion of internet users relative to the total population.
			\item Compute the correlation between the proportion of internet users and the number of cybercrime incidents.
			\item Generate visualizations such as scatter plots, time series graphs,
				and distribution maps to illustrate the relationship between internet access and cybercrime.
		\end{itemize}
