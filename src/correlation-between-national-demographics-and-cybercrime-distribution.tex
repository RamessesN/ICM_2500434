The distribution of cybercrime is closely tied to national demographic statistics,
with the number of cybercrime incidents showing a positive correlation with several key factors.
In this section, we explore the relationship between cybercrime and four primary demographic indicators:
the proportion of internet users in a country,
the country's GDP, and the proportion of the population with higher education.
By analyzing these factors,
we aim to uncover patterns and correlations that can provide insights into the drivers of cybercrime
and inform the development of more targeted and effective cybersecurity policies.
\subsection{Data Preprocessing}\label{subsec:data-preprocessing} %5.1
    To analyze the correlation between national demographics and cybercrime distribution,
    we first preprocess the relevant data.
    The demographic indicators—internet user penetration\cite{it-net-user-zs}, 
    GDP\cite{ny-gdp-mktp-cd} and 
    the proportion of the population with higher education\cite{se-ter-enrr}
    —are obtained from the World Bank's official website.
    Additionally, we utilize the annual cybercrime incident data for each country,
    which was previously processed in our earlier analysis.
    By integrating these datasets,
    we ensure a comprehensive foundation for examining the relationship between demographic factors and cybercrime trends.

    \subsubsection{Data Integration and Cleaning} %5.1.1
        For each dataset, we filter the data to include only the years from 2010 to 2022.
        After filtering,
        we handle missing values by allowing a maximum missing value proportion of 20\% for each country's data.
        Missing values within this threshold are filled using linear interpolation.
        Any data points that remain missing after interpolation
        are removed to ensure the integrity and reliability of the dataset.
        This preprocessing step ensures that our analysis is based on a consistent and high-quality dataset.

\subsection{} %5.2
    \subsubsection{Data Consolidation} %5.2.1
        After preprocessing the individual datasets, we integrate the internet user data and the cybercrime incident data.
        This is achieved by performing an inner join on the two datasets using country codes and years as the matching keys.
        The inner join ensures that only the countries and years present in both datasets are retained,
        resulting in a combined dataset where each entry corresponds to a specific country and year with complete data
        for both internet user penetration and cybercrime incidents.
        This step is crucial for ensuring the accuracy and consistency of our subsequent analysis.

    \subsubsection{Logarithmic Transformation} %5.2.2
        To address the skewness in the distribution of cybercrime incident counts,
        we apply a logarithmic transformation to the data.
        As described in Subsection~\ref{subsec:building-the-hotspot-map},
        we use the \textit{log1p} transformation, which computes the natural logarithm of \(1 + x\),
        where \(x\) is the original cybercrime count.
        This transformation reduces the impact of extreme values and makes the data more symmetric,
        bringing it closer to a normal distribution.
        By applying this transformation, we ensure that the data is better suited for statistical analysis and modeling.

    \subsubsection{Spearman Correlation Analysis} %5.2.3
        To quantify the relationship between those data and the number of cybercrime incidents,
        we calculate the Spearman correlation coefficient.
        This non-parametric measure assesses the strength and direction of the monotonic relationship between two variables.
        The Spearman correlation coefficient (\(\rho\)) and its associated \(p\)-value are computed,
        with the \(p\)-value used to determine the statistical significance of the correlation.
        A \(p\)-value smaller than \(1 \times 10^{-10}\) indicates an extremely strong statistical significance.
        The Spearman correlation coefficient is calculated as follows:
        \[ \rho = 1 - \frac{6 \sum d_i^2}{n(n^2 - 1)} \]
        where:
        \begin{itemize}
            \item \(\rho\) is the Spearman rank correlation coefficient,
            \item \(d_i\) is the difference between the ranks of each pair of observations, and
            \item \(n\) is the number of observations.
        \end{itemize}
