The distribution of cybercrime is closely tied to national demographic statistics,
with the number of cybercrime incidents showing a positive correlation with several key factors.
In this section, we explore the relationship between cybercrime and four primary demographic indicators:
the proportion of internet users in a country,
the country's GDP, and the proportion of the population with higher education.
By analyzing these factors,
we aim to uncover patterns and correlations that can provide insights into the drivers of cybercrime
and inform the development of more targeted and effective cybersecurity policies.
\subsection{Data Preprocessing}\label{subsec:data-preprocessing} %5.1
    To analyze the correlation between national demographics and cybercrime distribution,
    we first preprocess the relevant data.
    The demographic indicators—internet user penetration\cite{it-net-user-zs}, 
    GDP\cite{ny-gdp-mktp-cd} and 
    the proportion of the population with higher education\cite{se-ter-enrr}
    —are obtained from the World Bank's official website.
    Additionally, we utilize the annual cybercrime incident data for each country,
    which was previously processed in our earlier analysis.
    By integrating these datasets,
    we ensure a comprehensive foundation for examining the relationship between demographic factors and cybercrime trends.
