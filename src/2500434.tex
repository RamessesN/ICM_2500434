%%%%%%%%%%%%%%%%%%%%%%%%%%%%%%%%%%%%%%%%
%% MCM/ICM LaTeX Template %%
%% 2025 MCM/ICM %%
%%%%%%%%%%%%%%%%%%%%%%%%%%%%%%%%%%%%%%%%

% Format
\documentclass[12pt]{article}
\usepackage{geometry}
\geometry{left=1in, right=0.75in, top=1in, bottom=1in}

\newcommand{\Problem}{F}
\newcommand{\Team}{2500434}

\usepackage{newtxtext}
\usepackage{amsmath,amssymb,amsthm}
\usepackage{newtxmath}
\usepackage[pdftex]{graphicx}
\usepackage{subfig}
\usepackage{xcolor}
\usepackage{fancyhdr}
\usepackage{enumitem}
\usepackage{titlesec}
\usepackage{tocloft}
\usepackage{lastpage}

% Page Settings
\pagestyle{fancy}
\lhead{Team \# \Team}
\rhead{Page \thepage\ of \pageref{LastPage}}

% Mathematical env.
\newtheorem{theorem}{Theorem}
\newtheorem{corollary}[theorem]{Corollary}
\newtheorem{lemma}[theorem]{Lemma}
\newtheorem{definition}{Definition}

% Format of table of contents
\renewcommand{\cftsecleader}{\cftdotfill{\cftdotsep}}
\renewcommand{\contentsname}{\begin{center} INDEX \end{center}}

% Format of subsection and subsubsection
\titleformat{\section}{\bfseries\raggedright\Large}{\thesection}{1em}{}
\titleformat{\subsection}{\bfseries\raggedright}{\thesubsection}{1em}{}{}
\titleformat{\subsubsection}{\bfseries}{\hspace*{2em}\thesubsubsection}{1em}{}{}

% CoverPage Design
\begin{document}
\graphicspath{{.}}  % Place the graphic file in the same directory as the main document
\DeclareGraphicsExtensions{.pdf, .jpg, .tif, .png}
\thispagestyle{empty}
\vspace*{-16ex}
\centerline{
\begin{tabular}{*3{c}}
	\parbox[t]{0.3\linewidth}{\begin{center}\textbf{Problem Chosen}\\ \Large \textcolor{red}{\Problem}\end{center}}
	& \parbox[t]{0.3\linewidth}{\begin{center}\textbf{2025\\ MCM/ICM\\ Summary Sheet}\end{center}}
	& \parbox[t]{0.3\linewidth}{\begin{center}\textbf{Team Control Number}\\ \Large \textcolor{red}{\Team}\end{center}}	\\
	\hline
\end{tabular}}

% TODO: Summary
% 1. 12-point Times New Roman font ;
% 2. pdf naming format: 2500434.pdf ;
% 3. No name of school, advisor, or team members .
\begin{center}
Hello Summary!
\end{center}

\newpage
\tableofcontents
\newpage

\section{Introduction}\label{sec:introduction}
%! Author = Yilin
%! Date = 2025/1/26

\subsection{Background}\label{subsec:background} %1.1
	In the digital age, the speed and scale of global connectivity have reached unprecedented heights.
	However, with technological advancements, \textbf{cybercrime} has also become increasingly complex and diverse.
	These crimes pose significant threats and challenges to personal privacy, corporate assets, national security, and social stability.
	The transnational and covert nature of cybercrime makes it difficult to address effectively.
	Attackers often exploit legal differences and technical vulnerabilities across countries to evade accountability.
	Additionally, many businesses and institutions, concerned about their reputation and commercial interests,
	often choose \textbf{not to publicly report} cyberattacks, opting instead to pay ransoms or handle incidents privately.
	This further exacerbates the hidden nature of cybercrime.
	Developed countries, with their highly digitized economic and social structures, are often prime targets for cybercrime, while
	developing countries face their own unique challenges.

	To address this global challenge, countries have introduced national cybersecurity \textbf{policies}
	aimed at enhancing their defensive capabilities through legal, technical, and organizational cooperation.
	The \textbf{effectiveness} of these policies varies significantly across nations, and these differences may be closely related to factors
	such as policy design, enforcement, technological infrastructure, education levels, economic development, and internet penetration rates.
	In this context, understanding which factors make certain countries' cybersecurity policies more effective has become a critical issue.
	By analyzing the global distribution of cybercrime, national cybersecurity policies, and their outcomes,
	we can identify which policies and laws are particularly effective in preventing, prosecuting, and mitigating cybercrime.
	This data-driven analysis not only helps countries improve their cybersecurity policies
	but also provides valuable insights for global cybersecurity cooperation.
\subsection{Restatement of the problem}\label{subsec:restatement-of-the-problem} %1.2
	We are required to identify patterns
	that can inform the data-driven development and refinement of national cybersecurity policies and laws,
	which focused on those that have demonstrated effectiveness.
	Our goal is to develop a theory explaining what constitutes a strong national cybersecurity policy and then
	support it with a data-driven analysis.
	Several key aspects are listed below:
	\begin{itemize}
		\item Patterns in cybercrime distributed worldwide.
		\item Assessment to effectiveness of National Cybersecurity Policies.
		\item Correlation between national demographics and our distribution analysis.
	\end{itemize}
\subsection{Analysis of problems}\label{subsec:analysis-of-problems} %1.3
	We have divided each problem into several different steps:
	\subsubsection{Cybercrime distribution across the world} %1.3.1
		\begin{itemize}
			\item Process the JSON files published on VCDB .
			\item Use GCI as the primary indicator to assess countries with disproportionately high levels of cybercrime.
			\item Group countries based on GCI Tier measure, and
				visually represent the distribution of cybercrime in each group using heatmaps.
		\end{itemize}
	\subsubsection{Effective policy or law analytical approach} %1.3.2
		\begin{itemize}
			\item Identify a representative country from each of the T1 to T5 national clusters and
				collect cybersecurity-related policies enacted by these countries.
			\item Plot time-series line graphs depicting the trends of cybercrime over time and
				analyze which policies have been effective in curbing criminal activities.
			\item Conduct a targeted analysis of specific policy models' effectiveness in certain countries,
				focusing on particular indicators.
			\item Integrate temporal factors and national contexts to comprehensively evaluate the impact of policies.
		\end{itemize}
	\subsubsection{National demographics correlation} %1.3.3
		\begin{itemize}
			\item For each country, preprocess the annual proportion of internet users relative to the total population.
			\item Compute the correlation between the proportion of internet users and the number of cybercrime incidents.
			\item Generate visualizations such as scatter plots, time series graphs,
				and distribution maps to illustrate the relationship between internet access and cybercrime.
		\end{itemize} %1

\section{Symbol and Assumptions}\label{sec:symbol-and-assumptions}
%! Author = Yilin
%! Date = 2025/1/27

\subsection{Symbol Description}\label{subsec:symbol-description} %2.1
\begin{tabular}{cl}
    \hline
    \textbf{Symbol} & \textbf{Description} \\
    \hline
    $D_i$                     & Cybercrime distribution in each country. \\
    $P_i$                     & Population of each country. \\
    $y$                       & Transformed value using \( y = \log(1 + x) \). \\
    $\mathbb{E}[Crime_t]$     & Poisson regression model predicted value. \\
    $\beta_0$                 & Intercept term. \\
    $\beta_1,\beta_2,\beta_3$ & Regression coefficients. \\
    $Bill_{t-k}$              & Lagged variable. \\
    $K$                       & Greatest lagged number. \\
    $\rho$                    & Spearman rank correlation coefficient \\
    $d_i$                     & Difference between the ranks of each pair of observations \\
    $n$                       & Number of observations \\
    \hline
\end{tabular}

\bigskip

\noindent
\begin{tabular}{cl}
    \hline
    \textbf{Abbreviation} & \textbf{Full Form} \\
    \hline
    GCI   & Global Cybersecurity Index\cite{gci-2024} \\
    VERIS & the Vocabulary for Event Recording and Incident Sharing\cite{veris} \\
    VCDB  & the VERIS Community Database\cite{vcdb} \\
    GDP   & Gross Domestic Product \\
    \hline
\end{tabular}
\subsection{Assumption}\label{subsec:assumption} %2.2
\begin{itemize}
    \item Countries with a population below 5\% are excluded from the consideration of cybercrime distribution because
        a small change of number could bring a significant difference to statistical analysis.
    \item In the statistical analysis of the global distribution of cybercrime,
        factors such as national population growth, internet access, wealth levels, and education levels
        are assumed to have no significant impact on the quantitative distribution of cybercrime incidents.
        This study hypothesizes that the distribution of cybercrime can be more intuitively understood by focusing solely on the number of incidents,
        independent of these socio-economic variables.
        This assumption is based on all available data recorded since the inception of cybercrime statistics,
        aiming to isolate the distribution patterns of cybercrime from other potential influencing factors.
    \item The impact of newly enacted laws or policies on cybercrime is not instantaneous;
        there is a time lag before their effects become evident.
\end{itemize}
 %2

\section{A Data-Driven Model for Global Cybercrime Hotspot Mapping}\label{sec:a-data-driven-model-for-global-cybercrime-hotspot-mapping}
Despite the continuous evolution of national cybercrime since the inception of data collection,
along with changes in policies, legal frameworks, and population demographics,
we can create a global cybercrime hotspot map by leveraging crime data recorded by VERIS over the years.
This not only facilitates the analysis of cybercrime volumes by country
but also allows for fitting the data against policy and population variables to assess their influence on cybercrime trends.
\subsection{Cybercrime distribution across the globe}\label{subsec:Building-the-hotspot-map} %3.1
	We made use of a world-wide map to represent all cybercrime occurred around the world.
	In the map, the color filled in each country represents the total number of cybercrime incidents recorded since the beginning of the statistics.
	The color gradient, ranging from dark blue to dark red, corresponds to eight severity levels (1 to 8).
	Countries marked in blue indicate a low frequency of cybercrime incidents, while those marked in red represent a high density of such incidents.
	For instance, the United States, where the VERIS concept was first proposed, has the highest number of recorded incidents (7,236),
	whereas many other countries have only 1 or 2 recorded incidents.
	To address this significant disparity in data distribution, we applied a logarithmic transformation to the data using the formula
	\[ y=\log(1+x) \]
	where x here represents $D_i$.
	This transformation was implemented using the function
	\[ np.log1p() \]
	in Python to ensure computational precision and stability, particularly for small values.
	The final results are visualized in Figure 1.
	\begin{figure}[htbp]
		\centering
		\includegraphics[width=1\linewidth]{./rsrc/Crime_distribution}
		\caption{Crime distribution}\label{fig:crime-distribution}
	\end{figure}
\subsection{High-prevalence regions}\label{subsec:high-prevalence-regions} %3.2
	We obtained population data $P_i$ for various countries over recent decades from the World Bank Group's website.
	Simultaneously, we processed data from the VCDB to tabulate the annual number of cybercrime incidents $D_i$ for each country from 2000 to 2025.
	However, due to discrepancies in the specific countries reported by the World Bank Group and those listed in the VCDB,
	we had to exclude certain countries to ensure that only those appearing in both datasets were retained.
	Ultimately, 109 countries were included in the model.
	To represent the average number of cybercrime incidents per capita,
	we calculated the ratio $D_i/P_i$ for each year from 2000 to 2025.
	Since the resulting values were too small for practical analysis,
	we scaled them by a factor of $10^{8}$ to express the data as the number of cybercrime incidents per 100 million people,
	denoted as $mD/P_i$:
	\[ mD/P_i = \frac{D_i}{P_i} \times 10^{8} \]

	According to the GCI (Global Cybersecurity Index) standards, countries are classified into five tiers, denoted as T1 to T5.
	We used this classification as the basis for clustering analysis,
	dividing the 109 countries into five groups based on the percentiles published on the GCI website:
	the top 10\%, the next 20\%, the following 25\%, the subsequent 25\%, and the bottom 20\%.
	For each group, the annual average of $mD/P_i$ (the number of cybercrime incidents per 100 million people) was calculated.
	To visualize the results, we constructed a 3D clustering heatmap of cybercrime trends,
	where the x-axis represents the five tiers (T1 to T5), the y-axis represents the time span from 2000 to 2025, and the z-axis represents the average $mD/P_i$ values.
	This visualization is presented in Figure 2.
	\begin{figure}[htbp]
		\centering
		\includegraphics[width=0.8\linewidth]{./rsrc/3D_with_Spaced_Projection}
		\caption{3D with Spaced Projection}\label{fig:3D_with_Spaced_Projection}
	\end{figure}
\subsection{Other Cybercrime Incidents}\label{subsec:other-cybercrime-incedents} % 3.3
	Using additional data obtained from the VCDB,
	we constructed heatmaps on a global scale based on the number of successful cybercrimes, thwarted cybercrimes, and reported cybercrimes, respectively.
	Due to the disproportionately high volume of data from the United States,
	we applied the same logarithmic transformation (\( y = \log(1 + x) \)) as in Figure~\ref{fig:crime-distribution} for consistency,
	where $x$ represents successful attacks, thwarted attacks, and reported attacks,
	resulting in the three sub-figures presented in Figure~\ref{fig:other-cybercrime-incidents}.

	In sub-figure (a), the number of successful attacks closely aligns with the total number of attacks in most countries.
	For instance, the United States recorded 7,189 successful attacks out of 7,236 total attacks,
	yielding a success rate of \( \frac{7189}{7236} \approx 99.35\% \).
	Similarly, the United Kingdom reported 569 successful attacks out of 574 total attacks,
	with a success rate of \( \frac{569}{574} \approx 99.13\% \).

	In contrast, countries with lower attack volumes did not show significant differences between the total number of attacks and the number of successful attacks,
	indicating that almost every attempted attack was successful.

	In sub-figure (b), only the United States and Canada reported thwarted attack cases, with 6 and 2 instances, respectively.

	In sub-figure (c), the number of successfully reported attacks and the number of countries involved were significantly higher than in sub-figure (b).
	This suggests that while many attacks were successful, a portion of them were detected and reported.
	\begin{figure}[htbp]
		\centering
		\subfloat[Successful Cybercrime Incidents]{
			\includegraphics[width=1\linewidth]{./rsrc/Crime_Successful_distribution}
		}\\
		\subfloat[Mitigated Cybercrime Attempts]{
			\includegraphics[width=0.45\linewidth]{./rsrc/Crime_NearMiss_distribution}
		}\hfill
		\subfloat[Reported Cybercrime Incidents]{
			\includegraphics[width=0.45\linewidth]{./rsrc/Crime_Suspected_distribution}
		}\\
		\caption{Other Cybercrime Incidents}\label{fig:other-cybercrime-incidents}
	\end{figure} %3

\section{The establishment and solution of problem 2 model}\label{sec:the-establishment-and-solution-of-problem-2-model} %4
	To identify the effectiveness of national cybersecurity policies,
	it is essential to analyze the correlation between the implementation of these policies and the subsequent trends in cybercrime.
	By examining the distribution of cybercrimes and comparing it with the timing and content of various national policies,
	we can discern patterns that highlight which measures are particularly effective or ineffective.
	This analysis will focus on key metrics such as the reduction in
	cybercrime incidents, the success rate of prosecutions, and the overall resilience of national cybersecurity infrastructures.
	Through this data-driven approach,
	we aim to provide actionable insights for the development and refinement of cybersecurity policies.
	\subsection{Selection of Representative Centroid Countries}\label{subsec:selection-of-representative centroid-countries} %4.1
		Having constructed a clustering model to categorize countries into five clusters (T1 to T5) based on GCI and other relevant metrics,
		we now proceed to analyze the effectiveness of cybersecurity policies within each cluster.
		To ensure a representative and data-driven analysis,
		we will select one central country from each cluster that meets the following criteria:
		\begin{itemize}
			\item \textbf{Representativeness:}
				The selected country should typify the overall characteristics of its cluster,
				reflecting the general trends and patterns observed within that group.
			\item \textbf{Data Availability:}
				The country must have sufficient historical data on cybersecurity policies and legislation enacted over the past two decades,
				allowing for a comprehensive analysis of policy impacts.
		\end{itemize}
		By focusing on these representative countries,
		we aim to draw meaningful insights into the effectiveness of various cybersecurity policies and laws,
		which can then be generalized to other countries within the same cluster.

		To identify the representative country for each cluster,
		we first calculate the average Global Cybersecurity Index (GCI) for each cluster.
		The average GCI, denoted as \(\overline{GCI}\), is computed as follows:
		\[ \overline{GCI} = \frac{\sum_{i=1}^{n} GCI_i}{n} \]
		where \(n\) is the number of countries in the cluster.
		Next, we compute the absolute deviation of each country's GCI from the cluster average:
		\[|GCI_i - \overline{GCI}|\] .
		The country with the smallest deviation is considered the most representative of its cluster.
		After this initial selection, we further filter out countries with insufficient or incomplete legal and policy documentation.

		Through this process, we identify the following representative countries for each cluster:
		\begin{itemize}
			\item \textbf{T1: United States}
				~\cite{
					congress-website,
					nist-website,
					dhs-website,
					sec-website,
					whitehouse-website,
					investigatory-powers-act-2016,
					ncsc-uk,
					telecom-security-act-2021,
					uk-cyber-security-requirements-2024,
					uk-cybersecurity-timeline-2024}
			\item \textbf{T2: Japan}
				~\cite{
					it-basic-law-japan,
					ppc-legal-japan,
					nisc-japan,
					mofa-japan,
					japan-law-translation,
					cs-strategy-2015-japan,
					cs-strategy-2018-japan,
					telecom-business-act-japan,
					cs-strategy-2021-japan}
			\item \textbf{T3: China}
				~\cite{
					international-cybercrime,
					cybersecurity-law-china,
					internet-censorship-china,
					china-data-security-regulations,
					cryptography-law-china}
			\item \textbf{T4: Costa Rica}
				~\cite{
					costa-rica-cybersecurity-strategy,
					costa-rica-pop-up}
			\item \textbf{T5: Namibia}
				~\cite{
					namibia-pop-up,
					namibia-digital-odyssey,
					namibia-cybersecurity-strategy}
		\end{itemize}

\section{The establishment and solution of problem 3 model}\label{sec:the-establishment-and-solution-of-problem-3-model} %5
	\subsection{Shit, bro*3}\label{subsec:shit-bro*3} %5.1
	
\section{Future expected data}\label{sec:future-expected-data} %6

\section{Advantages \& Disadvantages}\label{sec:advantages-&-disadvantages} %7

\bibliographystyle{unsrt}
\bibliography{references} %8

\section*{Appendix}\label{sec:appendix} %9


\end{document}