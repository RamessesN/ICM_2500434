%%%%%%%%%%%%%%%%%%%%%%%%%%%%%%%%%%%%%%%%
%% MCM/ICM LaTeX Template %%
%% 2025 MCM/ICM %%
%%%%%%%%%%%%%%%%%%%%%%%%%%%%%%%%%%%%%%%%

% Format
\documentclass[12pt]{article}
\usepackage{geometry}
\geometry{left=1in, right=0.75in, top=1in, bottom=1in}

\newcommand{\Problem}{F}
\newcommand{\Team}{2500434}

\usepackage{newtxtext}
\usepackage{amsmath,amssymb,amsthm}
\usepackage{newtxmath}
\usepackage[pdftex]{graphicx}
\usepackage{subfig}
\usepackage{xcolor}
\usepackage{fancyhdr}
\usepackage{enumitem}
\usepackage{titlesec}
\usepackage{tocloft}
\usepackage{lastpage}
\usepackage[hidelinks]{hyperref}
\usepackage{changepage}

% Page Settings
\pagestyle{fancy}
\lhead{Team \# \Team}
\rhead{Page \thepage\ of \pageref{LastPage}}

% Mathematical env.
\newtheorem{theorem}{Theorem}
\newtheorem{corollary}[theorem]{Corollary}
\newtheorem{lemma}[theorem]{Lemma}
\newtheorem{definition}{Definition}

% Format of table of contents
\renewcommand{\cftsecleader}{\cftdotfill{\cftdotsep}}
\renewcommand{\contentsname}{\begin{center} INDEX \end{center}}

% Format of subsection and subsubsection
\titleformat{\section}{\bfseries\raggedright\Large}{\thesection}{1em}{}
\titleformat{\subsection}{\bfseries\raggedright}{\thesubsection}{1em}{}{}
\titleformat{\subsubsection}{\bfseries}{\hspace*{2em}\thesubsubsection}{1em}{}{}

% CoverPage Design
\begin{document}
\graphicspath{{.}}  % Place the graphic file in the same directory as the main document
\DeclareGraphicsExtensions{.pdf, .jpg, .tif, .png}
\thispagestyle{empty}
\vspace*{-16ex}
\centerline{
\begin{tabular}{*3{c}}
	\parbox[t]{0.3\linewidth}{\begin{center}\textbf{Problem Chosen}\\ \Large \textcolor{red}{\Problem}\end{center}}
	& \parbox[t]{0.3\linewidth}{\begin{center}\textbf{2025\\ MCM/ICM\\ Summary Sheet}\end{center}}
	& \parbox[t]{0.3\linewidth}{\begin{center}\textbf{Team Control Number}\\ \Large \textcolor{red}{\Team}\end{center}}	\\
	\hline
\end{tabular}}

% Summary %
\begin{center}
	\fontsize{16}{19}\selectfont \textbf{Modeling Cybercrime Governance: When Policies Meet the Attackers} \\
	\bigskip
	\fontsize{16}{19}\selectfont \textbf{Summary} \\
\end{center}
\begin{adjustwidth}{2em}{2em}
	\hspace*{1.5em} In recent years, especially since 2012, the number of cyber crimes in various countries has risen sharply.
	Developed countries, led by the United States and Europe, are more seriously affected by cyberattacks,
	and this trend may cause national data leakage and cause huge economic losses.
	Therefore, models are needed to quantify the impact of the cybercrime situation in each country under various factors.

	By the size of the $GCI$ indicator, we divided the countries into five gradients.
	The typical countries in each division are selected according to the variance and the amount of data.
	The typical countries based on this division have universality in data,
	which means that the conclusions obtained from the separate detailed analysis of the typical countries can also be applied to other countries in the same echelon.

	By filtering and refining the $json$ data information in the $VCDB$ repository,
	we establish a data-driven model to solve the annual cyber crime situation around the world, and display it in the form of heat maps.
	As for the processing of refined indicators (successful, thwarted, reported, prosecuted),
	we also first filter the data, and modify the screening range to Success, Reported, Near Miss and False positive.
	Then the corresponding data-driven model is established and displayed with the heat map.

	For the global cybercrime situation under the influence of law and policy,
	we first obtain the total population of each country in the world, filter and process the data, and then obtain the annual average crime rate of each country.
	We then collected all the cybersecurity laws and regulations from 2000 to 2023 for a typical country in different categories,
	in order to build a large data-driven model to correlate legal policies and crime rates for a typical country.
	At the same time, the corresponding line chart can be drawn to find out which bills in which years of typical countries have immediate effects,
	that is, they play an obvious role in the control of cyber crime in that year.
	Next, we analyze the law rate policy and typical national network security management through $Poisson$ regression.
	Because the timeliness of legal policies should be considered.

	To analyze the correlation between the distribution of cybercrime and national demographic data
	(such as Internet use, per capita wealth level, education level, etc.) indicators,
	we first collected and processed the relevant data on various refined indicators of typical countries in the past 20 years,
	and then used $TOPSIS$ based on entropy weight method to judge the correlation.
	Then calculate the $Spearman$ coefficient to quantitatively assess the correlation between the refinement indicators and cybercrime,
	so as to determine which specific demographic data refinement indicators have the greatest impact on dealing with cybercrime.

	Considering the above aspects, we can control the network crime rate to a great extent,
	so as to make the national network public security system more perfect and stable.
\end{adjustwidth}

\newpage
\tableofcontents
\newpage

\section{Introduction}\label{sec:introduction}
%! Author = Yilin
%! Date = 2025/1/26

\subsection{Background}\label{subsec:background} %1.1
	In the digital age, the speed and scale of global connectivity have reached unprecedented heights.
	However, with technological advancements, \textbf{cybercrime} has also become increasingly complex and diverse.
	These crimes pose significant threats and challenges to personal privacy, corporate assets, national security, and social stability.
	The transnational and covert nature of cybercrime makes it difficult to address effectively.
	Attackers often exploit legal differences and technical vulnerabilities across countries to evade accountability.
	Additionally, many businesses and institutions, concerned about their reputation and commercial interests,
	often choose \textbf{not to publicly report} cyberattacks, opting instead to pay ransoms or handle incidents privately.
	This further exacerbates the hidden nature of cybercrime.
	Developed countries, with their highly digitized economic and social structures, are often prime targets for cybercrime, while
	developing countries face their own unique challenges.

	To address this global challenge, countries have introduced national cybersecurity \textbf{policies}
	aimed at enhancing their defensive capabilities through legal, technical, and organizational cooperation.
	The \textbf{effectiveness} of these policies varies significantly across nations, and these differences may be closely related to factors
	such as policy design, enforcement, technological infrastructure, education levels, economic development, and internet penetration rates.
	In this context, understanding which factors make certain countries' cybersecurity policies more effective has become a critical issue.
	By analyzing the global distribution of cybercrime, national cybersecurity policies, and their outcomes,
	we can identify which policies and laws are particularly effective in preventing, prosecuting, and mitigating cybercrime.
	This data-driven analysis not only helps countries improve their cybersecurity policies
	but also provides valuable insights for global cybersecurity cooperation.
\subsection{Restatement of the problem}\label{subsec:restatement-of-the-problem} %1.2
	We are required to identify patterns
	that can inform the data-driven development and refinement of national cybersecurity policies and laws,
	which focused on those that have demonstrated effectiveness.
	Our goal is to develop a theory explaining what constitutes a strong national cybersecurity policy and then
	support it with a data-driven analysis.
	Several key aspects are listed below:
	\begin{itemize}
		\item Patterns in cybercrime distributed worldwide.
		\item Assessment to effectiveness of National Cybersecurity Policies.
		\item Correlation between national demographics and our distribution analysis.
	\end{itemize}
\subsection{Analysis of problems}\label{subsec:analysis-of-problems} %1.3
	We have divided each problem into several different steps:
	\subsubsection{Cybercrime distribution across the world} %1.3.1
		\begin{itemize}
			\item Process the JSON files published on VCDB .
			\item Use GCI as the primary indicator to assess countries with disproportionately high levels of cybercrime.
			\item Group countries based on GCI Tier measure, and
				visually represent the distribution of cybercrime in each group using heatmaps.
		\end{itemize}
	\subsubsection{Effective policy or law analytical approach} %1.3.2
		\begin{itemize}
			\item Identify a representative country from each of the T1 to T5 national clusters and
				collect cybersecurity-related policies enacted by these countries.
			\item Plot time-series line graphs depicting the trends of cybercrime over time and
				analyze which policies have been effective in curbing criminal activities.
			\item Conduct a targeted analysis of specific policy models' effectiveness in certain countries,
				focusing on particular indicators.
			\item Integrate temporal factors and national contexts to comprehensively evaluate the impact of policies.
		\end{itemize}
	\subsubsection{National demographics correlation} %1.3.3
		\begin{itemize}
			\item For each country, preprocess the annual proportion of internet users relative to the total population.
			\item Compute the correlation between the proportion of internet users and the number of cybercrime incidents.
			\item Generate visualizations such as scatter plots, time series graphs,
				and distribution maps to illustrate the relationship between internet access and cybercrime.
		\end{itemize} %1

\section{Symbol and Assumptions}\label{sec:symbol-and-assumptions}
%! Author = Yilin
%! Date = 2025/1/27

\subsection{Symbol Description}\label{subsec:symbol-description} %2.1
\begin{tabular}{cl}
    \hline
    \textbf{Symbol} & \textbf{Description} \\
    \hline
    $D_i$                     & Cybercrime distribution in each country. \\
    $P_i$                     & Population of each country. \\
    $y$                       & Transformed value using \( y = \log(1 + x) \). \\
    $\mathbb{E}[Crime_t]$     & Poisson regression model predicted value. \\
    $\beta_0$                 & Intercept term. \\
    $\beta_1,\beta_2,\beta_3$ & Regression coefficients. \\
    $Bill_{t-k}$              & Lagged variable. \\
    $K$                       & Greatest lagged number. \\
    $\rho$                    & Spearman rank correlation coefficient \\
    $d_i$                     & Difference between the ranks of each pair of observations \\
    $n$                       & Number of observations \\
    \hline
\end{tabular}

\bigskip

\noindent
\begin{tabular}{cl}
    \hline
    \textbf{Abbreviation} & \textbf{Full Form} \\
    \hline
    GCI   & Global Cybersecurity Index\cite{gci-2024} \\
    VERIS & the Vocabulary for Event Recording and Incident Sharing\cite{veris} \\
    VCDB  & the VERIS Community Database\cite{vcdb} \\
    GDP   & Gross Domestic Product \\
    \hline
\end{tabular}
\subsection{Assumption}\label{subsec:assumption} %2.2
\begin{itemize}
    \item Countries with a population below 5\% are excluded from the consideration of cybercrime distribution because
        a small change of number could bring a significant difference to statistical analysis.
    \item In the statistical analysis of the global distribution of cybercrime,
        factors such as national population growth, internet access, wealth levels, and education levels
        are assumed to have no significant impact on the quantitative distribution of cybercrime incidents.
        This study hypothesizes that the distribution of cybercrime can be more intuitively understood by focusing solely on the number of incidents,
        independent of these socio-economic variables.
        This assumption is based on all available data recorded since the inception of cybercrime statistics,
        aiming to isolate the distribution patterns of cybercrime from other potential influencing factors.
    \item The impact of newly enacted laws or policies on cybercrime is not instantaneous;
        there is a time lag before their effects become evident.
\end{itemize}
 %2

\section{A Data-Driven Model for Global Cybercrime Hotspot Mapping}
\label{sec:a-data-driven-model-for-global-cybercrime-hotspot-mapping}
Despite the continuous evolution of national cybercrime since the inception of data collection,
along with changes in policies, legal frameworks, and population demographics,
we can create a global cybercrime hotspot map by leveraging crime data recorded by VERIS over the years.
This not only facilitates the analysis of cybercrime volumes by country
but also allows for fitting the data against policy and population variables to assess their influence on cybercrime trends.
\subsection{Cybercrime distribution across the globe}\label{subsec:Building-the-hotspot-map} %3.1
	We made use of a world-wide map to represent all cybercrime occurred around the world.
	In the map, the color filled in each country represents the total number of cybercrime incidents recorded since the beginning of the statistics.
	The color gradient, ranging from dark blue to dark red, corresponds to eight severity levels (1 to 8).
	Countries marked in blue indicate a low frequency of cybercrime incidents, while those marked in red represent a high density of such incidents.
	For instance, the United States, where the VERIS concept was first proposed, has the highest number of recorded incidents (7,236),
	whereas many other countries have only 1 or 2 recorded incidents.
	To address this significant disparity in data distribution, we applied a logarithmic transformation to the data using the formula
	\[ y=\log(1+x) \]
	where x here represents $D_i$.
	This transformation was implemented using the function
	\[ np.log1p() \]
	in Python to ensure computational precision and stability, particularly for small values.
	The final results are visualized in Figure 1.
	\begin{figure}[htbp]
		\centering
		\includegraphics[width=1\linewidth]{./rsrc/Crime_distribution}
		\caption{Crime distribution}\label{fig:crime-distribution}
	\end{figure}
\subsection{High-prevalence regions}\label{subsec:high-prevalence-regions} %3.2
	We obtained population data $P_i$ for various countries over recent decades from the World Bank Group's website.
	Simultaneously, we processed data from the VCDB to tabulate the annual number of cybercrime incidents $D_i$ for each country from 2000 to 2025.
	However, due to discrepancies in the specific countries reported by the World Bank Group and those listed in the VCDB,
	we had to exclude certain countries to ensure that only those appearing in both datasets were retained.
	Ultimately, 109 countries were included in the model.
	To represent the average number of cybercrime incidents per capita,
	we calculated the ratio $D_i/P_i$ for each year from 2000 to 2025.
	Since the resulting values were too small for practical analysis,
	we scaled them by a factor of $10^{8}$ to express the data as the number of cybercrime incidents per 100 million people,
	denoted as $mD/P_i$:
	\[ mD/P_i = \frac{D_i}{P_i} \times 10^{8} \]

	According to the GCI (Global Cybersecurity Index) standards, countries are classified into five tiers, denoted as T1 to T5.
	We used this classification as the basis for clustering analysis,
	dividing the 109 countries into five groups based on the percentiles published on the GCI website:
	the top 10\%, the next 20\%, the following 25\%, the subsequent 25\%, and the bottom 20\%.
	For each group, the annual average of $mD/P_i$ (the number of cybercrime incidents per 100 million people) was calculated.
	To visualize the results, we constructed a 3D clustering heatmap of cybercrime trends,
	where the x-axis represents the five tiers (T1 to T5), the y-axis represents the time span from 2000 to 2025, and the z-axis represents the average $mD/P_i$ values.
	This visualization is presented in Figure 2.
	\begin{figure}[htbp]
		\centering
		\includegraphics[width=0.8\linewidth]{./rsrc/3D_with_Spaced_Projection}
		\caption{3D with Spaced Projection}\label{fig:3D_with_Spaced_Projection}
	\end{figure}
\subsection{Other Cybercrime Incidents}\label{subsec:other-cybercrime-incedents} % 3.3
	Using additional data obtained from the VCDB,
	we constructed heatmaps on a global scale based on the number of successful cybercrimes, thwarted cybercrimes, and reported cybercrimes, respectively.
	Due to the disproportionately high volume of data from the United States,
	we applied the same logarithmic transformation (\( y = \log(1 + x) \)) as in Figure~\ref{fig:crime-distribution} for consistency,
	where $x$ represents successful attacks, thwarted attacks, and reported attacks,
	resulting in the three sub-figures presented in Figure~\ref{fig:other-cybercrime-incidents}.

	In sub-figure (a), the number of successful attacks closely aligns with the total number of attacks in most countries.
	For instance, the United States recorded 7,189 successful attacks out of 7,236 total attacks,
	yielding a success rate of \( \frac{7189}{7236} \approx 99.35\% \).
	Similarly, the United Kingdom reported 569 successful attacks out of 574 total attacks,
	with a success rate of \( \frac{569}{574} \approx 99.13\% \).

	In contrast, countries with lower attack volumes did not show significant differences between the total number of attacks and the number of successful attacks,
	indicating that almost every attempted attack was successful.

	In sub-figure (b), only the United States and Canada reported thwarted attack cases, with 6 and 2 instances, respectively.

	In sub-figure (c), the number of successfully reported attacks and the number of countries involved were significantly higher than in sub-figure (b).
	This suggests that while many attacks were successful, a portion of them were detected and reported.
	\begin{figure}[htbp]
		\centering
		\subfloat[Successful Cybercrime Incidents]{
			\includegraphics[width=1\linewidth]{./rsrc/Crime_Successful_distribution}
		}\\
		\subfloat[Mitigated Cybercrime Attempts]{
			\includegraphics[width=0.45\linewidth]{./rsrc/Crime_NearMiss_distribution}
		}\hfill
		\subfloat[Reported Cybercrime Incidents]{
			\includegraphics[width=0.45\linewidth]{./rsrc/Crime_Suspected_distribution}
		}\\
		\caption{Other Cybercrime Incidents}\label{fig:other-cybercrime-incidents}
	\end{figure} %3

\section{Policy Identification and Analysis}\label{sec:policy-identification-and-analysis}
To identify the effectiveness of national cybersecurity policies,
it is essential to analyze the correlation between the implementation of these policies and the subsequent trends in cybercrime.
By examining the distribution of cybercrimes and comparing it with the timing and content of various national policies,
we can discern patterns that highlight which measures are particularly effective or ineffective.
This analysis will focus on key metrics such as the reduction in
cybercrime incidents, the success rate of prosecutions, and the overall resilience of national cybersecurity infrastructures.
Through this data-driven approach,
we aim to provide actionable insights for the development and refinement of cybersecurity policies.
\subsection{Selection of Representative Centroid Countries}\label{subsec:selection-of-representative centroid-countries} %4.1
Having constructed a clustering model to categorize countries into five clusters (T1 to T5) based on GCI and other relevant metrics,
we now proceed to analyze the effectiveness of cybersecurity policies within each cluster.
To ensure a representative and data-driven analysis,
we will select one central country from each cluster that meets the following criteria:
\begin{itemize}
    \item \textbf{Representativeness:}
    The selected country should typify the overall characteristics of its cluster,
    reflecting the general trends and patterns observed within that group.
    \item \textbf{Data Availability:}
    The country must have sufficient historical data on cybersecurity policies and legislation enacted over the past two decades,
    allowing for a comprehensive analysis of policy impacts.
\end{itemize}
By focusing on these representative countries,
we aim to draw meaningful insights into the effectiveness of various cybersecurity policies and laws,
which can then be generalized to other countries within the same cluster.

\subsubsection*{Implementation Steps} %4.1.1
    To identify the representative country for each cluster,
    we first calculate the average GCI for each cluster.
    The average GCI, denoted as \(\overline{GCI}\), is computed as follows:
    \[ \overline{GCI} = \frac{\sum_{i=1}^{n} GCI_i}{n} \]
    where \(n\) is the number of countries in the cluster.
    Next, we compute the absolute deviation of each country's GCI from the cluster average:
    \[ \{ a|a= |GCI_i - \overline{GCI}| \} \]
    where \(a\) is the approach to the average \( \overline{GCI} \).
    The country with the smallest deviation is considered the most representative of its cluster.
    After this initial selection, we further filter out countries with insufficient or incomplete legal and policy documentation.

    Through this process, we identify the following representative countries for each cluster with their references:
    \begin{itemize}
        \item \textbf{T1: United States}
        ~\cite{
            congress-website,
            nist-website,
            dhs-website,
            sec-website,
            whitehouse-website,
            investigatory-powers-act-2016,
            ncsc-uk,
            telecom-security-act-2021,
            uk-cyber-security-requirements-2024,
            uk-cybersecurity-timeline-2024}
        \item \textbf{T2: Japan}
        ~\cite{
            it-basic-law-japan,
            ppc-legal-japan,
            nisc-japan,
            mofa-japan,
            japan-law-translation,
            cs-strategy-2015-japan,
            cs-strategy-2018-japan,
            telecom-business-act-japan,
            cs-strategy-2021-japan}
        \item \textbf{T3: China}
        ~\cite{
            international-cybercrime,
            cybersecurity-law-china,
            internet-censorship-china,
            china-data-security-regulations,
            cryptography-law-china}
        \item \textbf{T4: Costa Rica}
        ~\cite{
            costa-rica-cybersecurity-strategy,
            costa-rica-pop-up}
        \item \textbf{T5: Namibia}
        ~\cite{
            namibia-pop-up,
            namibia-digital-odyssey,
            namibia-cybersecurity-strategy}
    \end{itemize}

\subsubsection*{Visualizing Policy Impact Over Time} %4.1.2
    With the selected representative countries,
    we proceed to visualize the impact of cybersecurity policies on cybercrime trends.
    For each country, we plot a line graph where
    the x-axis represents time (from 2000 to 2023) and the y-axis represents the annual number of cybercrime incidents.
    To highlight the influence of policy implementations,
    we mark the data points corresponding to years in which cybersecurity policies or laws were enacted with an orange color.
    This visualization is presented in Figure\ref{fig:representative-policy}.

    This allows us to preliminarily assess the effectiveness of the policies.
    Specifically:
    \begin{itemize}
        \item A downward trend in the line graph following the implementation of a policy (marked in orange)
        suggests that the policy may have been effective in reducing cybercrime.
        \item An upward or unchanged trend, on the other hand,
        may indicate that the policy was ineffective or had unintended consequences.
    \end{itemize}

    This initial analysis provides a broad overview of the impact of various policies and
    helps identify patterns that warrant further investigation.
    It also serves as a foundation for more detailed analysis of specific policies, guiding future research directions.

    \begin{figure}[htbp]
        \centering
        \subfloat[US (T1)]{
            \includegraphics[width=0.45\linewidth]{../rsrc/policies/(T1)US_policy}
        }\hfill
        \subfloat[Japan (T2)]{
            \includegraphics[width=0.45\linewidth]{../rsrc/policies/(T2)Japan_policy}
        }\\
        \subfloat[China (T3)]{
            \includegraphics[width=0.3\linewidth]{../rsrc/policies/(T3)China_policy}
        }\hfill
        \subfloat[Costa Rica (T4)]{
            \includegraphics[width=0.3\linewidth]{../rsrc/policies/(T4)Costa_Rica_policy}
        }\hfill
        \subfloat[Namibia (T5)]{
            \includegraphics[width=0.3\linewidth]{../rsrc/policies/(T5)Namibia_policy}
        }\\
        \caption{Representative countries}\label{fig:representative-policy}
    \end{figure}

\subsubsection*{Categorizing Policies Based on Effectiveness} %4.1.3
    From the line graphs, we categorize the enacted policies into three sets
    based on the average cybercrime metrics in the years following their implementation compared to the year of enactment.
    Drawing inspiration from fuzzy set theory,
    we assign a value of \(1\) to policies that are effective,
    \(-1\) to those with the opposite effect,
    and values between \(-1\) and \(1\) to policies with varying degrees of impact.
    Specifically:

    \begin{itemize}
        \item \textbf{Effective Policies \( (Value \to 1) \):}
            These are policies where the average cybercrime metrics in the years following enactment
            are lower than those in the year of enactment.
            Examples include:
            \begin{itemize}
                \item National Institute of Standards and Technology (NIST) Cybersecurity Framework (2014)
                \item National Cyber Security Centre (NCSC) Establishment (2016)
                \item Investigatory Powers Act (2016)
                \item Cybersecurity Strategy (2013)
                \item Telecommunications Business Act Amendments (2019)
                \item Cryptography Law of the People’s Republic of China (2019)
            \end{itemize}

        \item \textbf{Counterproductive Policies \( (Value \to -1) \):}
            These are policies where the average cybercrime metrics in the years following enactment
            are higher than those in the year of enactment.
            Examples include:
            \begin{itemize}
                \item Cyber Incident Reporting for Critical Infrastructure Act (CIRCIA) (2022)
                \item Quantum Computing Cybersecurity Preparedness Act (2022)
                \item Strengthening American Cybersecurity Act (2022)
                \item CHIPS and Science Act (2022)
                \item Cybersecurity Strategy (2018)
                \item Cybersecurity Law of the People’s Republic of China (2017)
            \end{itemize}

        \item \textbf{Neutral or Mixed-Impact Policies (Values around 0):}
            These are policies where
            the average cybercrime metrics show no significant change, or the impact is ambiguous.
            Representative examples include:
            \begin{itemize}
                \item Policy A
                \item Policy B
                \item Policy C
                \item (Additional policies to be listed)
            \end{itemize}
    \end{itemize}

    This categorization provides a structured framework for analyzing the effectiveness of cybersecurity policies and
    serves as a basis for further investigation into the factors that contribute to their success or failure.

 %4

\section{Correlation Between National Demographics and Cybercrime Distribution}
\label{sec:correlation-between-national-demographics-and-cybercrime-distribution}
The distribution of cybercrime is closely tied to national demographic statistics,
with the number of cybercrime incidents showing a positive correlation with several key factors.
In this section, we explore the relationship between cybercrime and four primary demographic indicators:
the proportion of internet users in a country,
the country's GDP, and the proportion of the population with higher education.
By analyzing these factors,
we aim to uncover patterns and correlations that can provide insights into the drivers of cybercrime
and inform the development of more targeted and effective cybersecurity policies.
\subsection{Data Preprocessing}\label{subsec:data-preprocessing} %5.1
    To analyze the correlation between national demographics and cybercrime distribution,
    we first preprocess the relevant data.
    The demographic indicators—internet user penetration\cite{it-net-user-zs}, 
    GDP\cite{ny-gdp-mktp-cd} and 
    the proportion of the population with higher education\cite{se-ter-enrr}
    —are obtained from the World Bank's official website.
    Additionally, we utilize the annual cybercrime incident data for each country,
    which was previously processed in our earlier analysis.
    By integrating these datasets,
    we ensure a comprehensive foundation for examining the relationship between demographic factors and cybercrime trends.
 %5

\section{Future Work}\label{sec:future-work} %6
%! Author = Mei
%! Date = 2025/1/27

To enhance the model's robustness and applicability, several improvements are proposed.
\begin{itemize}
    \item Incorporating \textbf{time-series} or \textbf{panel data analysis} would capture dynamic effects,
        such as the lagged impacts of policy changes on crime rates.
    \item Integrating \textbf{machine learning} methods like random forests
        could better handle nonlinear relationships and complex variable interactions,
        addressing limitations of linear models.
    \item Developing a simplified \textbf{visualization tool} to translate complex model results
        into easily interpretable charts would improve accessibility for policymakers,
        facilitating more informed decision-making.
\end{itemize}
These steps aim to address current limitations and expand the model's practical utility.


\section{Strengths \& Weaknesses}\label{sec:strength-&-weakness}
%! Author = Yilin
%! Date = 2025/1/27

\subsection{Strengths}\label{subsec:strengths} %7.1
    \begin{itemize}
        \item By filtering the data and drawing the global heat map of cybercrime distribution,
            the global distribution of cybercrime can be visualized.
            Then, by filtering the data according to specific indicators
            (Confirmed, Near miss, Reported, Prosecuted)
            and drawing the corresponding global heat map,
            the distribution of these indicators can be visualized.
        \item Through processing the filtered data and drawing a line chart,
            it can be more intuitive to see which laws and policies have benefits for the national cybercrime rate control;
            However, the timeliness of legal policies cannot be rationally judged, that is,
            whether the law has an impact on the future cannot be directly considered.
        \item Introduction of Poisson model: Because the number of cybercrimes is a typical non-negative integer data,
            Poisson regression analysis can avoid the continuity assumption of ordinary linear regression,
            and can model and explain the nonlinear association between the number of bills and the crime rate and zero inflation data.
            Furthermore, the regression coefficients can be used for policy evaluation by calculating the incidence ratio $IRR$ .
        \item Introduction of time hysteresis effect:
            The introduction of time lag effect can reflect the delay of the policy coming into effect, that is,
            the release of legal policies does not necessarily produce benefits at the time but will have potential impacts on the future.
            Then, the optimal lag period was selected by cross validation,
            and the key time node for the policy to be effective was clarified.
        \item The approach is data-driven and objective.
            The entropy weight method reduces subjective bias by determining weights based on data dispersion,
            while multiple linear regression quantifies the impact of factors on cybercrime rates, enhancing result credibility.
            A multi-method framework is employed: regression analysis identifies direct relationships, and
            $TOPSIS$ integrates multidimensional indicators like the $GCI$ index to rank countries by policy effectiveness, aiding decision-making.
            The model is interpretable, with regression coefficients and entropy weights clearly quantifying factor influences e.g.,
            a one-unit $GDP$ increase raises crime rates by 0.32.
            It also leverages existing $GCI$ index for validation, avoiding redundancy.
            Additionally, techniques like residual analysis and $VIF$ detection address multi-collinearity and data bias, improving robustness.
            Hierarchical regression and interaction terms help mitigate confounding effects,
            such as separating $GDP$ and internet penetration influences.
    \end{itemize}

\subsection{Weaknesses}\label{subsec:weaknesses} %7.2
    \begin{itemize}
        \item Pure data screening and analysis need to be based on a huge amount of data.
            For these big countries in Europe and the United States, the annual amount of data is very considerable,
            so the effect is very prominent.
            But for some small countries in Africa or Asia,
            the difference in the distribution map of cybercrime is not very obvious because of the insufficient amount of data.
        \item Introduction of Poisson model: Over-discretization of the data can lead to bias in the model,
            and the model will only work for relationships that are roughly linear,
            and it will be less useful if the actual relationship is more complex (e.g., if there are interaction effects).
        \item The approach relies heavily on data quality.
            Underreporting of cybercrime may underestimate crime rates,
            while missing data of incomplete $GCI$ scores for developing nations introduces selection bias, limiting generalizability.
            Multi-collinearity remains a challenge; high correlations between $GDP$ and internet penetration can weaken regression results despite VIF and ridge regression adjustments.
            Complex interactions of education and internet usage are poorly captured by linear models.
            Finally, regression analysis only identifies correlations, not causation,
            and the lack of instrumental variables or experimental design limits causal inference.
    \end{itemize}

    Therefore, it is necessary to combine the data analysis and Poisson \& time lag effect model
    to obtain the income effect of legal policies on the national Internet crime rate.

 %7

\newpage
\section*{\begin{center} Memo \end{center}}\label{sec:memo}
%! Author = Mei
%! Date = 2025/1/27

\vspace*{-3em}
As global digitalization accelerates,
cybercrime has emerged as a transnational threat, with its costs and risks escalating worldwide.
However, the effectiveness of national cybersecurity policies remains inconsistent,
with many failing to adequately address evolving criminal tactics.
By analyzing global datasets—including cybercrime trends, demographic factors, economic indicators,
and the International Telecommunication Union’s (ITU) Global Cybersecurity Index (GCI)
—our research aims to identify core patterns of effective policies and
provide actionable recommendations tailored to national contexts.


Our analysis reveals systemic patterns that demands coordinated action across three dimensions:

While economic prosperity (measured by GDP) correlates with increased exposure to financially
motivated attacks (e.g., ransomware, banking fraud), education investment acts as a counterweight:
nations achieving a 10\% increase in tertiary education enrollment
observe an 8\% decline in citizen victimization rates.
Conversely, advanced cybersecurity infrastructure—while critical for threat detection—
paradoxically elevates a country’s attractiveness to state-sponsored or organized cybercrime groups,
as seen in the U.S.\ and Germany.

Effective cybersecurity strategies hinge on two synergistic pillars:
legal-operational alignment and adaptive governance frameworks.
First, harmonizing domestic cyber laws with international cooperation mechanisms
—such as streamlined cross-border data-sharing treaties—enables nations to achieve
20--30\% higher rates of both crime reporting and prosecution success (ITU GCI data).
Second, biennial revisions of technical standards, as opposed to static policies,
have proven instrumental in curbing cybercrime growth rates by 15\%,
demonstrating the urgency of institutionalizing policy agility.

The cybersecurity landscape is further complicated by systemic blind spots.
Corporate non-disclosure practices leave an estimated 30\% of cyber incidents unreported,
distorting risk assessments and perpetuating reactive policymaking;
meanwhile, the rapid digitalization of developing economies—particularly in Southeast Asia and Africa
—has outpaced institutional capacity-building, rendering these regions critical vulnerabilities.
In 2023 alone, 55\% of cross-border attacks exploited infrastructure gaps in these emerging markets,
underscoring the need for targeted global capacity-sharing initiatives.


Our analysis identifies three actionable levers to disrupt cybercrime's transnational trajectory,
grounded in successful national models and global benchmarks:

Countries allocating at least 5\% of education budgets to cybersecurity literacy programs
—particularly targeting high-risk sectors like finance and healthcare
—reduce phishing success rates by 12--18\% (EU case studies).
However, technological investments (e.g., AI-driven threat detection)
must be coupled with mandatory incident reporting laws,
as seen in Japan’s 2023 policy overhaul, which increased attack disclosure by 40\% within six months.

Real-time sharing of digital forensics data (e.g., attacker infrastructure fingerprints)
between nations can shorten incident response times by up to 65\%,
as demonstrated by the ASEAN Cybersecurity Pact.
The ITU’s GCI highlights pioneers like Singapore and Estonia,
where regional response centers reduced cross-border ransomware damage
costs by \$2.1 billion annually through coordinated threat neutralization.

Proactive publication of national cyber resilience metrics
—such as attack types, victim demographics, and policy outcomes—creates market incentives for corporate compliance.
Brazil’s 2022 transparency initiative, linked to a public dashboard monitoring critical infrastructure risks,
spurred 92\% adherence to revised cybersecurity standards among energy sector firms.
 %8
\newpage

\bibliographystyle{unsrt}
\bibliography{references} %9

\newpage
\section*{\centerline{\underline{Report on Use of AI}}}
\begin{enumerate}
    \item OpenAI ChatGPT( Jan 25, 2025 version, ChatGPT-4o(Internet Search) )
    \begin{tabular}{ll}
        Query1: & Please give me more US cybersecurity laws and regulations from 2000 to 2023.
        Output: &
    \end{tabular}

    \item
    \begin{tabular}{ll}
        Query1: &
        Output: &
    \end{tabular}

    \item
    \begin{tabular}{ll}
        Query1: &
        Output: &
    \end{tabular}

    \item
    \begin{tabular}{ll}
        Query1: &
        Output: &
    \end{tabular}

    \item
    \begin{tabular}{ll}
        Query1: &
        Output: &
    \end{tabular}

    \item
    \begin{tabular}{ll}
        Query1: &
        Output: &
    \end{tabular}

    \item
    \begin{tabular}{ll}
        Query1: &
        Output: &
    \end{tabular}

    \item
    \begin{tabular}{ll}
        Query1: &
        Output: &
    \end{tabular}

    \item
    \begin{tabular}{ll}
        Query1: &
        Output: &
    \end{tabular}

    \item
    \begin{tabular}{ll}
        Query1: &
        Output: &
    \end{tabular}

\end{enumerate}


\end{document}
