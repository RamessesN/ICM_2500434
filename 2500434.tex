%%%%%%%%%%%%%%%%%%%%%%%%%%%%%%%%%%%%%%%%
%% MCM/ICM LaTeX Template %%
%% 2025 MCM/ICM %%
%%%%%%%%%%%%%%%%%%%%%%%%%%%%%%%%%%%%%%%%

% Format
\documentclass[12pt]{article}
\usepackage{geometry}
\geometry{left=1in, right=0.75in, top=1in, bottom=1in}

\newcommand{\Problem}{F}
\newcommand{\Team}{2500434}

\usepackage{newtxtext}
\usepackage{amsmath,amssymb,amsthm}
\usepackage{newtxmath}
\usepackage[pdftex]{graphicx}
\usepackage{xcolor}
\usepackage{fancyhdr}
\usepackage{enumitem}
\usepackage{titlesec}
\usepackage{tocloft}
\usepackage{lastpage}

% Page Settings
\pagestyle{fancy}
\lhead{Team \# \Team}
\rhead{Page \thepage\ of \pageref{LastPage}}

% Mathematical env.
\newtheorem{theorem}{Theorem}
\newtheorem{corollary}[theorem]{Corollary}
\newtheorem{lemma}[theorem]{Lemma}
\newtheorem{definition}{Definition}

% Format of table of contents
\renewcommand{\cftsecleader}{\cftdotfill{\cftdotsep}}
\renewcommand{\contentsname}{\begin{center} INDEX \end{center}}

% Format of subsection and subsubsection
\titleformat{\section}{\bfseries\centering\Large}{\thesection}{1em}{}
\titleformat{\subsection}{\bfseries\raggedright}{\thesubsection}{1em}{}{}
\titleformat{\subsubsection}{\bfseries}{\hspace*{2em}\thesubsubsection}{1em}{}{}

% CoverPage Design
\begin{document}
\graphicspath{{.}}  % Place the graphic file in the same directory as the main document
\DeclareGraphicsExtensions{.pdf, .jpg, .tif, .png}
\thispagestyle{empty}
\vspace*{-16ex}
\centerline{
\begin{tabular}{*3{c}}
	\parbox[t]{0.3\linewidth}{\begin{center}\textbf{Problem Chosen}\\ \Large \textcolor{red}{\Problem}\end{center}}
	& \parbox[t]{0.3\linewidth}{\begin{center}\textbf{2025\\ MCM/ICM\\ Summary Sheet}\end{center}}
	& \parbox[t]{0.3\linewidth}{\begin{center}\textbf{Team Control Number}\\ \Large \textcolor{red}{\Team}\end{center}}	\\
	\hline
\end{tabular}}

% TODO: Summary
% 1. 12-point Times New Roman font ;
% 2. pdf naming format: 2500434.pdf ;
% 3. No name of school, advisor, or team members .
\begin{center}
Hello Summary!
\end{center}

\newpage
\tableofcontents
\newpage

\section{Introduction}\label{sec:introduction} %1
	\subsection{Background}\label{subsec:background} %1.1
		In the digital age, the speed and scale of global connectivity have reached unprecedented heights.
		However, with technological advancements, \( \textbf{cybercrime} \) has also become increasingly complex and diverse.
		These crimes pose significant threats and challenges to personal privacy, corporate assets, national security, and social stability.
		The transnational and covert nature of cybercrime makes it difficult to address effectively.
		Attackers often exploit legal differences and technical vulnerabilities across countries to evade accountability.
		Additionally, many businesses and institutions, concerned about their reputation and commercial interests,
		often choose not to publicly report cyberattacks, opting instead to pay ransoms or handle incidents privately.
		This further exacerbates the hidden nature of cybercrime.
		Developed countries, with their highly digitized economic and social structures, are often prime targets for cybercrime, while
		developing countries face their own unique challenges.

		To address this global challenge,
		countries have introduced national cybersecurity \( \textbf{policies} \) aimed at enhancing their defensive capabilities through legal, technical, and organizational cooperation.
		The \( \textbf{effectiveness} \) of these policies varies significantly across nations, and
		these differences may be closely related to factors such as policy design, enforcement, technological infrastructure, education levels, economic development, and internet penetration rates.

		In this context, understanding which factors make certain countries' cybersecurity policies more effective has become a critical issue.
		By analyzing the global distribution of cybercrime, national cybersecurity policies, and their outcomes,
		we can identify which policies and laws are particularly effective in preventing, prosecuting, and mitigating cybercrime.
		This data-driven analysis not only helps countries improve their cybersecurity policies but also provides valuable insights for global cybersecurity cooperation.
		% \subsubsection{Shit*1, bro!} %1.1.1
	\subsection{Restatement of the problem}\label{subsec:restatement-of-the-problem} %1.2
		We are asked to give a
	\subsection{Analysis of problems}\label{subsec:analysis-of-problems} %1.3
	
\section{Symbol and Assumptions}\label{sec:symbol-and-assumptions} %2
	\subsection{Symbol Description}\label{subsec:symbol-description} %2.1
	
\section{The establishment and solution of problem 1 model}\label{sec:the-establishment-and-solution-of-problem-1-model} %3
	\subsection{Shit, bro*1}\label{subsec:shit-bro*1} %3.1
	
\section{The establishment and solution of problem 2 model}\label{sec:the-establishment-and-solution-of-problem-2-model} %4
	\subsection{Shit, bro*2}\label{subsec:shit-bro*2} %4.1
	
\section{The establishment and solution of problem 3 model}\label{sec:the-establishment-and-solution-of-problem-3-model} %5
	\subsection{Shit, bro*3}\label{subsec:shit-bro*3} %5.1
	
\section{Future expected data}\label{sec:future-expected-data} %6

\section{Advantages \& Disadvantages}\label{sec:advantages-&-disadvantages} %7

\section{References}\label{sec:references} %8

\section{Appendix}\label{sec:appendix} %9


\end{document}